\lstset{ %
  language=R,                     % the language of the code
  basicstyle=\footnotesize,       % the size of the fonts that are used for the code
  numbers=left,                   % where to put the line-numbers
  numberstyle=\tiny\color{black},  % the style that is used for the line-numbers
  stepnumber=1,                   % the step between two line-numbers. If it's 1, each line
                                  % will be numbered
  numbersep=5pt,                  % how far the line-numbers are from the code
  backgroundcolor=\color{white},  % choose the background color. You must add \usepackage{color}
  showspaces=false,               % show spaces adding particular underscores
  showstringspaces=false,         % underline spaces within strings
  showtabs=false,                 % show tabs within strings adding particular underscores
  frame=single,                   % adds a frame around the code
  rulecolor=\color{black},        % if not set, the frame-color may be changed on line-breaks within not-black text (e.g. commens (green here))
  tabsize=2,                      % sets default tabsize to 2 spaces
  captionpos=b,                   % sets the caption-position to bottom
  breaklines=true,                % sets automatic line breaking
  breakatwhitespace=false,        % sets if automatic breaks should only happen at whitespace
  title=\lstname,                 % show the filename of files included with \lstinputlisting;
                                  % also try caption instead of title
  keywordstyle=\color{blue},      % keyword style
  commentstyle=\color{green},   % comment style
  stringstyle=\color{blue},      % string literal style
  escapeinside={\%*}{*)},         % if you want to add a comment within your code
  morekeywords={*,...}            % if you want to add more keywords to the set
} 
\lstlistoflistings

\begin{lstlisting}[caption="Initial script for randomization"]
# calling library and write randomized experiment to R data file
library(FrF2)
plan <- FrF2(nruns=16,nfactors=3,alias.info=2,randomize=TRUE)
plan
save(plan,file="plan.Rda")
plan
rm(plan)
\end{lstlisting}

\begin{lstlisting}[caption="Main code"]
# loading randomized design of experiment from R data file
# Without saving the randomization, the experiment would be 
# near impossible to reproduce
library(FrF2)
library(MASS)
require(graphics)
load("plan.Rda")
plan
plan2 <-plan
summary(plan)

#adding response vector Y to the plan 
# observational data
y <- c(2.0,6.2,3.4,2.6,3.9,3.3,2.9,2.0,2.1,2.6,1.9,2.7,2.9,4.6,2.3,2.2)
ymod <- y^(-3.6)
plan <- add.response(plan,ymod)
plan

library(leaps)
regfitF=regsubsets(ymod~.,data=plan,method="forward")
sumregF <- summary(regfitF)
sumregF # best model includes all factors 

# regressing the plan including transformed 
# response vector fourth root
model1 <- lm(ymod~A+B+C+A:C+A:B,data=plan)
summary(model1)
plot(model1$fitted,rstudent(model1),pch=20)
plot(c(1:16),model1$residuals)
qqnorm(rstudent(model1),pch=20)
qqline(rstudent(model1))
boxcox(model1,lambda = seq(-5, 5))
anova(model1)
DanielPlot(model1)
MEPlot(model1)
IAPlot(model1)

lenth <- function(lmobj,alpha=0.05)
{
  abseffects <- abs(2*lmobj$coeff)[-1]
  medabseffects <- median(abseffects)
  medabseffects
  s0 <- 1.5*medabseffects
  keepeffects <- abseffects[abseffects< 2.5*s0]
  PSE <- 1.5*median(keepeffects)
  signlimit <-qt(1-alpha/2,length(abseffects)/3)*PSE
  return(list("PSE"=PSE,"Signlimit"=signlimit,"Number of sign"=sum(abseffects>signlimit)))
}

lenthres <- lenth(model1)
lenthres
# see that factors with effects larger than lenthres$Signlimit should be assumed
# to have effect different from zero
effects <- 2*model1$coeff
effects>lenthres$Signlimit

# How to make a relatively ok Paretoplot for the effects
barplot(sort(abs(effects[-1]),decreasing=FALSE),las=1,horiz=TRUE,cex.names=1.0)
abline(v=lenthres$Signlimit,col=2,lwd=2)
lenthres$Signlimit

\end{lstlisting}