Which factors do you think are relevant to the problem described above?
Do you expect an interaction between some of the factors?
Which levels should be used, and why do you think these are reasonable?
How can you control that the factors really are at the desired level?

The relevant factors to this problem are A: Hammer hand; B: Nail temperature and C:Nail surface. It is believed that these factors are as favorable for the estimation as possible \parencite[1]{tyssedal}.

All 3 factors are considered to be physical differences. For factor A it is difficult to imagine that it will be equally fast which ever hand used. For factor B it is unclear whether temperature does effect response. If it does, it may be thinkable that the combination of a hot nail, hammered with right hand and without threads will be the fastest. It is not immideately apparent that any two of the explanatory variables goes under the common conception of interactions \parencite{interactionswiki}. 

The levels are for A: left hand/ right hand; B: hot/ room tempered; C: threaded/ smooth. The A and C levels seem reasonable because they reflect everyday variations in carpenters work. Factor B is relevant because that factor is the key factor. Table \ref{tab:factors} gives an overview of the mentioned factors and their levels. 

\begin{table}[H]
    \centering
    \begin{tabular}{|c|c|c|c|}
        \hline
        & A  & B & C\\
        & Hand  & Temp. [C] & Shape \\
        \hline
         + & Right & 225 C & Round \\
         - & Left & 15C & Square \\
         \hline
    \end{tabular}
    \caption{Levels, factors and effect coding}
    \label{tab:factors}
\end{table}

The factors and their appropriate levels are very much controllable. From the video it will be clear which hammer hand and which nail surface are being used. Nail temperature is not visible on the video document, but very much so upon touch and heat radiation.